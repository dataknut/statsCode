\documentclass[]{article}
\usepackage{lmodern}
\usepackage{amssymb,amsmath}
\usepackage{ifxetex,ifluatex}
\usepackage{fixltx2e} % provides \textsubscript
\ifnum 0\ifxetex 1\fi\ifluatex 1\fi=0 % if pdftex
  \usepackage[T1]{fontenc}
  \usepackage[utf8]{inputenc}
\else % if luatex or xelatex
  \ifxetex
    \usepackage{mathspec}
  \else
    \usepackage{fontspec}
  \fi
  \defaultfontfeatures{Ligatures=TeX,Scale=MatchLowercase}
\fi
% use upquote if available, for straight quotes in verbatim environments
\IfFileExists{upquote.sty}{\usepackage{upquote}}{}
% use microtype if available
\IfFileExists{microtype.sty}{%
\usepackage{microtype}
\UseMicrotypeSet[protrusion]{basicmath} % disable protrusion for tt fonts
}{}
\usepackage[margin=1in]{geometry}
\usepackage{hyperref}
\hypersetup{unicode=true,
            pdftitle={Testing gt for tables},
            pdfauthor={Ben Anderson (Contact: b.anderson@soton.ac.uk, @dataknut)},
            pdfborder={0 0 0},
            breaklinks=true}
\urlstyle{same}  % don't use monospace font for urls
\usepackage{color}
\usepackage{fancyvrb}
\newcommand{\VerbBar}{|}
\newcommand{\VERB}{\Verb[commandchars=\\\{\}]}
\DefineVerbatimEnvironment{Highlighting}{Verbatim}{commandchars=\\\{\}}
% Add ',fontsize=\small' for more characters per line
\usepackage{framed}
\definecolor{shadecolor}{RGB}{248,248,248}
\newenvironment{Shaded}{\begin{snugshade}}{\end{snugshade}}
\newcommand{\KeywordTok}[1]{\textcolor[rgb]{0.13,0.29,0.53}{\textbf{#1}}}
\newcommand{\DataTypeTok}[1]{\textcolor[rgb]{0.13,0.29,0.53}{#1}}
\newcommand{\DecValTok}[1]{\textcolor[rgb]{0.00,0.00,0.81}{#1}}
\newcommand{\BaseNTok}[1]{\textcolor[rgb]{0.00,0.00,0.81}{#1}}
\newcommand{\FloatTok}[1]{\textcolor[rgb]{0.00,0.00,0.81}{#1}}
\newcommand{\ConstantTok}[1]{\textcolor[rgb]{0.00,0.00,0.00}{#1}}
\newcommand{\CharTok}[1]{\textcolor[rgb]{0.31,0.60,0.02}{#1}}
\newcommand{\SpecialCharTok}[1]{\textcolor[rgb]{0.00,0.00,0.00}{#1}}
\newcommand{\StringTok}[1]{\textcolor[rgb]{0.31,0.60,0.02}{#1}}
\newcommand{\VerbatimStringTok}[1]{\textcolor[rgb]{0.31,0.60,0.02}{#1}}
\newcommand{\SpecialStringTok}[1]{\textcolor[rgb]{0.31,0.60,0.02}{#1}}
\newcommand{\ImportTok}[1]{#1}
\newcommand{\CommentTok}[1]{\textcolor[rgb]{0.56,0.35,0.01}{\textit{#1}}}
\newcommand{\DocumentationTok}[1]{\textcolor[rgb]{0.56,0.35,0.01}{\textbf{\textit{#1}}}}
\newcommand{\AnnotationTok}[1]{\textcolor[rgb]{0.56,0.35,0.01}{\textbf{\textit{#1}}}}
\newcommand{\CommentVarTok}[1]{\textcolor[rgb]{0.56,0.35,0.01}{\textbf{\textit{#1}}}}
\newcommand{\OtherTok}[1]{\textcolor[rgb]{0.56,0.35,0.01}{#1}}
\newcommand{\FunctionTok}[1]{\textcolor[rgb]{0.00,0.00,0.00}{#1}}
\newcommand{\VariableTok}[1]{\textcolor[rgb]{0.00,0.00,0.00}{#1}}
\newcommand{\ControlFlowTok}[1]{\textcolor[rgb]{0.13,0.29,0.53}{\textbf{#1}}}
\newcommand{\OperatorTok}[1]{\textcolor[rgb]{0.81,0.36,0.00}{\textbf{#1}}}
\newcommand{\BuiltInTok}[1]{#1}
\newcommand{\ExtensionTok}[1]{#1}
\newcommand{\PreprocessorTok}[1]{\textcolor[rgb]{0.56,0.35,0.01}{\textit{#1}}}
\newcommand{\AttributeTok}[1]{\textcolor[rgb]{0.77,0.63,0.00}{#1}}
\newcommand{\RegionMarkerTok}[1]{#1}
\newcommand{\InformationTok}[1]{\textcolor[rgb]{0.56,0.35,0.01}{\textbf{\textit{#1}}}}
\newcommand{\WarningTok}[1]{\textcolor[rgb]{0.56,0.35,0.01}{\textbf{\textit{#1}}}}
\newcommand{\AlertTok}[1]{\textcolor[rgb]{0.94,0.16,0.16}{#1}}
\newcommand{\ErrorTok}[1]{\textcolor[rgb]{0.64,0.00,0.00}{\textbf{#1}}}
\newcommand{\NormalTok}[1]{#1}
\usepackage{longtable,booktabs}
\usepackage{graphicx,grffile}
\makeatletter
\def\maxwidth{\ifdim\Gin@nat@width>\linewidth\linewidth\else\Gin@nat@width\fi}
\def\maxheight{\ifdim\Gin@nat@height>\textheight\textheight\else\Gin@nat@height\fi}
\makeatother
% Scale images if necessary, so that they will not overflow the page
% margins by default, and it is still possible to overwrite the defaults
% using explicit options in \includegraphics[width, height, ...]{}
\setkeys{Gin}{width=\maxwidth,height=\maxheight,keepaspectratio}
\IfFileExists{parskip.sty}{%
\usepackage{parskip}
}{% else
\setlength{\parindent}{0pt}
\setlength{\parskip}{6pt plus 2pt minus 1pt}
}
\setlength{\emergencystretch}{3em}  % prevent overfull lines
\providecommand{\tightlist}{%
  \setlength{\itemsep}{0pt}\setlength{\parskip}{0pt}}
\setcounter{secnumdepth}{5}
% Redefines (sub)paragraphs to behave more like sections
\ifx\paragraph\undefined\else
\let\oldparagraph\paragraph
\renewcommand{\paragraph}[1]{\oldparagraph{#1}\mbox{}}
\fi
\ifx\subparagraph\undefined\else
\let\oldsubparagraph\subparagraph
\renewcommand{\subparagraph}[1]{\oldsubparagraph{#1}\mbox{}}
\fi

%%% Use protect on footnotes to avoid problems with footnotes in titles
\let\rmarkdownfootnote\footnote%
\def\footnote{\protect\rmarkdownfootnote}

%%% Change title format to be more compact
\usepackage{titling}

% Create subtitle command for use in maketitle
\providecommand{\subtitle}[1]{
  \posttitle{
    \begin{center}\large#1\end{center}
    }
}

\setlength{\droptitle}{-2em}

  \title{Testing gt for tables}
    \pretitle{\vspace{\droptitle}\centering\huge}
  \posttitle{\par}
    \author{Ben Anderson (Contact:
\href{mailto:b.anderson@soton.ac.uk}{\nolinkurl{b.anderson@soton.ac.uk}},
\texttt{@dataknut})}
    \preauthor{\centering\large\emph}
  \postauthor{\par}
      \predate{\centering\large\emph}
  \postdate{\par}
    \date{Last run at: 2019-05-10 16:33:44}

\usepackage{amsmath}
\usepackage{booktabs}
\usepackage{caption}
\usepackage{longtable}

\begin{document}
\maketitle

{
\setcounter{tocdepth}{2}
\tableofcontents
}
\section{Introduction}\label{introduction}

\texttt{kableExtra} makes great tables in html \& pdf but not so good in
MS Word. Sadly many of our co-authors still like to use MS Word. So we
need a Word-friendly way to make nice tables\ldots{}

\section{gt}\label{gt}

The \href{https://github.com/rstudio/gt}{gt package} might help.

\begin{Shaded}
\begin{Highlighting}[]
\NormalTok{df <-}\StringTok{ }\NormalTok{gtcars}
\NormalTok{dt <-}\StringTok{ }\KeywordTok{as.data.table}\NormalTok{(df) }\CommentTok{# cos we like data.tables}

\NormalTok{t <-}\StringTok{ }\KeywordTok{table}\NormalTok{(dt}\OperatorTok{$}\NormalTok{mfr, dt}\OperatorTok{$}\NormalTok{bdy_style)}

\KeywordTok{gt}\NormalTok{(t)}
\end{Highlighting}
\end{Shaded}

\captionsetup[table]{labelformat=empty,skip=1pt}

\begin{longtable}{rrr}
\toprule
Var1 & Var2 & Freq \\ 
\midrule
Acura & convertible & 0 \\ 
Aston Martin & convertible & 0 \\ 
Audi & convertible & 0 \\ 
Bentley & convertible & 0 \\ 
BMW & convertible & 0 \\ 
Chevrolet & convertible & 0 \\ 
Dodge & convertible & 0 \\ 
Ferrari & convertible & 2 \\ 
Ford & convertible & 0 \\ 
Jaguar & convertible & 0 \\ 
Lamborghini & convertible & 0 \\ 
Lotus & convertible & 0 \\ 
Maserati & convertible & 0 \\ 
McLaren & convertible & 0 \\ 
Mercedes-Benz & convertible & 1 \\ 
Nissan & convertible & 0 \\ 
Porsche & convertible & 1 \\ 
Rolls-Royce & convertible & 1 \\ 
Tesla & convertible & 0 \\ 
\bottomrule
\end{longtable}

So a gt object is clearly in long form. The examples all use pipes to
pass the table through dplyr stuff\ldots{}

\begin{Shaded}
\begin{Highlighting}[]
\CommentTok{# Define the start and end dates for the data range}
\NormalTok{start_date <-}\StringTok{ "2010-06-07"}
\NormalTok{end_date <-}\StringTok{ "2010-06-14"}

\CommentTok{# Create a gt table based on preprocessed}
\CommentTok{# `sp500` table data}
\NormalTok{sp500 }\OperatorTok
\StringTok{  }\NormalTok{dplyr}\OperatorTok{::}\KeywordTok{filter}\NormalTok{(date }\OperatorTok{>=}\StringTok{ }\NormalTok{start_date }\OperatorTok{&}\StringTok{ }\NormalTok{date }\OperatorTok{<=}\StringTok{ }\NormalTok{end_date) }\OperatorTok
\StringTok{  }\NormalTok{dplyr}\OperatorTok{::}\KeywordTok{select}\NormalTok{(}\OperatorTok{-}\NormalTok{adj_close) }\OperatorTok
\StringTok{  }\KeywordTok{gt}\NormalTok{() }\OperatorTok
\StringTok{  }\KeywordTok{tab_header}\NormalTok{(}
    \DataTypeTok{title =} \StringTok{"S&P 500"}\NormalTok{,}
    \DataTypeTok{subtitle =}\NormalTok{ glue}\OperatorTok{::}\KeywordTok{glue}\NormalTok{(}\StringTok{"\{start_date\} to \{end_date\}"}\NormalTok{)}
\NormalTok{  ) }\OperatorTok
\StringTok{  }\KeywordTok{fmt_date}\NormalTok{(}
    \DataTypeTok{columns =} \KeywordTok{vars}\NormalTok{(date),}
    \DataTypeTok{date_style =} \DecValTok{3}
\NormalTok{  ) }\OperatorTok
\StringTok{  }\KeywordTok{fmt_currency}\NormalTok{(}
    \DataTypeTok{columns =} \KeywordTok{vars}\NormalTok{(open, high, low, close),}
    \DataTypeTok{currency =} \StringTok{"USD"}
\NormalTok{  ) }\OperatorTok
\StringTok{  }\KeywordTok{fmt_number}\NormalTok{(}
    \DataTypeTok{columns =} \KeywordTok{vars}\NormalTok{(volume),}
    \DataTypeTok{suffixing =} \OtherTok{TRUE}
\NormalTok{  )}
\end{Highlighting}
\end{Shaded}

\captionsetup[table]{labelformat=empty,skip=1pt}

\begin{longtable}{lrrrrr}
\caption*{
\large S\&P 500\\ 
\small 2010-06-07 to 2010-06-14\\ 
} \\ 
\toprule
date & open & high & low & close & volume \\ 
\midrule
Mon, Jun 14, 2010 & $\text{\$}1,095.00$ & $\text{\$}1,105.91$ & $\text{\$}1,089.03$ & $\text{\$}1,089.63$ & $4.43B$ \\ 
Fri, Jun 11, 2010 & $\text{\$}1,082.65$ & $\text{\$}1,092.25$ & $\text{\$}1,077.12$ & $\text{\$}1,091.60$ & $4.06B$ \\ 
Thu, Jun 10, 2010 & $\text{\$}1,058.77$ & $\text{\$}1,087.85$ & $\text{\$}1,058.77$ & $\text{\$}1,086.84$ & $5.14B$ \\ 
Wed, Jun 9, 2010 & $\text{\$}1,062.75$ & $\text{\$}1,077.74$ & $\text{\$}1,052.25$ & $\text{\$}1,055.69$ & $5.98B$ \\ 
Tue, Jun 8, 2010 & $\text{\$}1,050.81$ & $\text{\$}1,063.15$ & $\text{\$}1,042.17$ & $\text{\$}1,062.00$ & $6.19B$ \\ 
Mon, Jun 7, 2010 & $\text{\$}1,065.84$ & $\text{\$}1,071.36$ & $\text{\$}1,049.86$ & $\text{\$}1,050.47$ & $5.47B$ \\ 
\bottomrule
\end{longtable}

\section{R environment}\label{r-environment}

Packages used:

\begin{itemize}
\tightlist
\item
  gt (Iannone, Cheng, and Schloerke 2019)
\end{itemize}

And:

\begin{itemize}
\tightlist
\item
  tidyverse - (Wickham 2017)
\item
  glue - ({\textbf{???}})
\item
  data.table - (Dowle et al. 2015)
\end{itemize}

\section*{References}\label{references}
\addcontentsline{toc}{section}{References}

\hypertarget{refs}{}
\hypertarget{ref-data.table}{}
Dowle, M, A Srinivasan, T Short, S Lianoglou with contributions from R
Saporta, and E Antonyan. 2015. \emph{Data.table: Extension of
Data.frame}. \url{https://CRAN.R-project.org/package=data.table}.

\hypertarget{ref-gt}{}
Iannone, Richard, Joe Cheng, and Barret Schloerke. 2019. \emph{Gt:
Easily Create Presentation-Ready Display Tables}.
\url{https://github.com/rstudio/gt}.

\hypertarget{ref-tidyverse}{}
Wickham, Hadley. 2017. \emph{Tidyverse: Easily Install and Load
'Tidyverse' Packages}.
\url{https://CRAN.R-project.org/package=tidyverse}.


\end{document}
